% !TEX root = main.tex
\usepackage{listings}
\usepackage{accsupp}
\usepackage[margin=12pt]{caption}

% LstListing Color
\definecolor{commentGreen}{rgb}{0,0.6,0}
\definecolor{numberGray}{rgb}{0.25,0.25,0.25}
\definecolor{stringMauve}{rgb}{0.08,0.52,0.1}
\definecolor{backgroundGray}{rgb}{0.98,0.98,0.98}

\lstset{ %
	abovecaptionskip=12pt,
	backgroundcolor=\color{backgroundGray},
	basewidth=0.5em,
	basicstyle=\small\ttfamily,
	breakatwhitespace=false,
	breaklines=true,
	captionpos=b,
	columns=flexible,
	commentstyle=\color{commentGreen},
	deletekeywords={...},
	escapeinside={(*@}{@*)},
	extendedchars=false,
	frame=lr,
	framesep=20pt,
	framerule=0pt,
	keepspaces=true,
	keywordstyle=\color{blue},
	language=Java,
	otherkeywords={my-component,export,@Component,@Directive,@Injectable,@Input,@Output,io-component,ee-node,ee-separator,ee-panel,ee-panel-header,ee-table},
	numbers=left,
	numbersep=5pt,
	numberstyle=\scriptsize\color{numberGray}\noncopynumber,
	rulecolor=\color{black},
	showspaces=false,
	showstringspaces=false,
	showtabs=false,
	stepnumber=1,
	stringstyle=\color{stringMauve},
	tabsize=4,
	title=\lstname,
	xleftmargin=0.5cm
}

\newcommand{\noncopynumber}[1]{
	\BeginAccSupp{method=escape,ActualText={}}
	#1
	\EndAccSupp{}
}

\renewcommand{\lstlistingname}{Codeausschnitt}