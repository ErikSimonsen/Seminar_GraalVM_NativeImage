% !TEX root = main.tex
\section{Einleitung}
\label{sec:einleitung}

\subsection{Motivation}
\label{subsec:motivation}
Softwareentwicklung hat sich in den letzten Jahren stark verändert, moderne Anwendungen setzen immer öfter auf Microservices und verschiedene Modelle des Cloud-Computings, wie
Serverless Computing. Statt große Anwendungsserver zu nutzen, werden dabei unabhängige Services deployed (Function as a Service, FaaS). Sobald die Workload steigt, also die Anzahl der Anfragen an den Service steigen,
werden von der jeweiligen Cloudplattform mehrere \textit{Worker} erstellt, welche die Anfragen bearbeiten. Bei der ersten Anfrage die ein neuer \textit{Worker} bearbeitet, muss zuerst die Laufzeitumgebung 
der jeweiligen Programmiersprache und die Konfiguration des Services initialisiert werden. Da solche Kaltstarts bei virtuellen Machinen häufig sehr langsam sind\footnote{Besonders bei Java VMs
 durch Ausführen von Class Loading, Profiling, dynamischer Kompilierung etc.} und Cloudanbieter nach Start- und Laufzeit einer Funktion abrechnen bietet sich die Nutzung von Java VMs nicht an.

\subsection{GraalVM}
\label{subsec:graalvm}

GraalVM ist ein Softwareprodukt von Oracle. Es wird als Ökosystem bezeichnet und die zentrale Komponente ist der \textit{GraalVM Compiler}, ein in Java geschriebener JIT(Just-in-time)-Compiler für Java, der durch neue Codeanalyse- und 
 Optimierungsmethoden erhebliche Performancevorteile bietet.
Darüber hinaus bietet GraalVM die Möglichkeit der polyglotten Programmierung mithilfe des Truffle-Frameworks und der Ausführung von nativem Code auf der JVM, durch einen LLVM Bitcode Interpreter.
Zudem bietet GraalVM die \textit{Native Image} Funktionalität, die im weiteren Verlauf dieser Arbeit thematisiert wird. \textit{Native Image} ermöglicht das ahead-of-time kompilieren von Java-Anwendungen in nativ ausführbare Dateien (executables).
Die Grundlage von GraalVM bietet das OpenJDK\parencite{GraalComponents}.\newline

 Darin enthalten ist, neben Komponenten \& Werkzeugen für die Entwicklung, auch die Java-Laufzeitumgebung (Java Runtime Environment, JRE), welche die HotSpot JVM enthält.
Die Hotspot JVM enthält zwei JIT-Compiler: C1, einen schnellen und nur leicht optimierenden Compiler, ursprünglich gedacht für Desktopanwendungen, und C2, einen aggressiv optimierenden Compiler für 
Serveranwendungen, bei denen es auf höchsten Durchsatz ankommt.
Über das JMVCI (Java Virtual Machine Compiler Interface, \parencite{OpenJDK243}) wird der GraalVM-Compiler in das OpenJDK integriert und ersetzt den Compiler C2 der HotSpot JVM. Entgegen des Namens ist GraalVM
also genau genommen keine eigene virtuelle Maschine, sondern nutzt die HotSpot JVM mit einem anderen JIT-Compiler.
\newpage
\subsection{NativeImage}
\label{subsec:nativeImage}

Die \textit{NativeImage}-Funktionalität wird bei GraalVM als optionale Komponente angeboten, und muss explizit installiert werden. 
Sie erlaubt die AOT-Kompilierung von Java-Anwendungen in nativ ausführbare Dateien (executables bzw. \textit{native images}).
Das executable läuft nicht auf der JVM und bezieht Funktionalitäten, wie Speicherverwaltung, Thread-Scheduling und Garbage Collection von einer, auf das notwendigste reduzierten, in Java geschriebenen virtuellen Machine
, genannt \textit{SubstrateVM}\parencite{GraalVMNativeImage}. Die \textit{SubstrateVM} wird, zusätzlich zum Anwendungscode, Bibliotheken und dem JDK in das \textit{native image} hineinkompiliert.
Weil das executable in sich abgeschlossen ist muss dabei die Closed-World Annahme erfüllt werden, das heißt jeglicher Bytecode muss zum Zeitpunkt der Kompilierung bzw. zur Build-Time bekannt und vorhanden sein, und 
zur Laufzeit dürfen keine neuen Klassen geladen bzw. Bytecode generiert werden. Ansonsten wäre ein Großteil der ausgeführten Optimierungen während der Build-Time nicht möglich.

Im Weiteren Verlauf der Arbeit wird der gesamte Build-Prozess erläutert, Performanceparameter zwischen einer Anwendung als \textit{native image} und auf der Hotspot JVM verglichen, sowie die Limitierungen des 
\textit{native image}-Ansatzes dargelegt.
