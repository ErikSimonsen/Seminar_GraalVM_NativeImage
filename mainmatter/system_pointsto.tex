\subsection{Points-to Analyse}
\label{subsec:pointsto}
Die Points-To (dt. Pointer-Analyse) ist eine Analysetechnik für statischen Code. Dabei wird ermittelt welche Pointer und Heap-Referenzen auf welche Variablen und Speicherbereiche zeigen.

Die Points-To Analysis (dt. Pointer-Analyse) ist eine Technik zur Berechnung der Menge von Objekten bzw. dessen Speicherbereichen, auf die eine Programm Variable zur Laufzeit zeigen kann\cite{Hind2001, Smaragdakis2015}. Da das zu analysierende Programm bzw. dessen kompletter Quellcode vorliegen muss, ist die Pointer-Analyse eine statische Analysetechnik.
