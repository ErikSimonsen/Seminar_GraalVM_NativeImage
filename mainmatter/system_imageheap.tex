\subsection{Image Heap}
\label{subsec:imageheap}

Die Ausführung des \textit{native image} erfolgt mit einem bereits, durch das Heap-Snapshotting zur Build-Time, gefüllten Java Heap,
dem \textit{image heap}. Das Starten des executables und das Erstellen von isolierten VM Instanzen (\textit{Isolates} in einem existierenden Prozess muss
schnell sein und wenig Speicheraufwand benötigen, damit für kurzfristige Aufgaben sofort mehrere VM-Instanzen gestartet werden können.
Um das zu gewährleisten, darf der image heap nicht für jede Instanz neu kopiert werden, sondern wird auf verschiedene Speicheradressen im selben Adressraum
abgebildet. Daher sind alle Referenzen relativ zum Start des \textit{image heaps}, sowohl von bereits im image heap vorhandenen Objekten, als auch von Objekten die 
erst zur Laufzeit allokiert werden. Dadurch kann der \textit{image heap} mehrere Male im Speicher des Adressraumes abgebildet werden, und erlaubt somit
schnelles Erstellen von Isolaten, da das Replizieren des \textit{image heaps} kein Kopieren erfordert. Zudem wird der Speicheraufwand durch das, auf den
image heap angewandte, Copy-On-Write Verfahren deutlich reduziert. Dabei erhalten alle Prozesse die auf die gleiche Speicherseite zugreifen, eine flache Kopie davon, also nur Zeiger die 
auf die tatsächliche Speicheradressen zeigen. Solange die Prozesse die Daten lediglich auslesen, ist es nicht nötig die Daten zu kopieren oder erneut im Hauptspeicher anzulegen. Alle Prozesse greifen also
auf das \grqq{}Original\grqq{} zu. Erst sobald ein Prozess Daten ändert (write), wird ein neuer Datenblock angelegt(copy) und die entsprechenden Referenzen angepasst \parencite{bovet2002understanding}.
Der \textit{image heap} wird vom \textit{garbage collector} als Wurzelzeiger (root pointer) behandelt, und wird daher nicht deallokiert.
