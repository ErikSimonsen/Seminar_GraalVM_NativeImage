\subsection{Initialisierungscode}
\label{subsec:initializationcode}
Wenn bei der Pointer-Analyse keine Typen mehr zu den Typlisten der Knoten hinzugefügt werden, wird der Initialisierungscode ausgeführt. Als Initialisierungscode werden Klasseninitialisierer und Callbacks durch die Feature-API von GraalVM eingestuft.
Im Klasseninitialisierer einer Klasse werden dessen statische Felder initialisiert. Standardmäßig werden alle Klassen erst zur Laufzeit initialisiert, der Entwickler kann jedoch, um die Performance zur Laufzeit zu optimieren, mit dem Befehl \textit{--initialize-at-build-time} eine Liste von Klassen angeben die bereits zur Buildtime initialisiert werden\parencite[Ab Version 19.0]{Wimmer2019Medium}.
Durch die Feature-API kann die Anwendung Callback-Funktionen, mit benutzerdefiniertem Code, registrieren die zur Buildtime, also vor/während/nach der Analysephase ausgeführt werden \parencite{GraalVM}. Das Besondere ist, dass der Initialisierungscode den Status der Pointer-Analyse abfragen kann. Es kann abgerufen werden, ob ein Typ, eine Methode oder ein Feld bereits als \glqq erreichbar\grqq{} markiert wurde.
