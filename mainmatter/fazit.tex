% !TEX root = main.tex
\section{Fazit}
\label{sec:fazit}

Ziel dieser Arbeit war es einen groben Überblick über das NativeImage-Modul von GraalVM zu geben, und die Funktionsweise, sowie die Systemkomponenten zu erläutern. 
Durch \textit{native image} wird ein leichtgewichtige, in sich geschlossene ausführbare Datei (executable) erzeugt. Dies geschieht indem die Points-To Analyse, das Ausführen
von Initialisierungscode und Heap-Snapshotting iterativ ausgeführt werden und anschließend die ahead-of-time Kompilierung erfolgt.

Durch die Initialisierung der Anwendung zur Build-Time und das Nutzen des \textit{image heaps} wird eine schnelle Startzeit, eine geringe Speichernutzung und eine gute Spitzenleistung
ermöglicht. Beim Vergleich populärer Java-Microservice Frameworks konnte im Durchschnitt eine um den Faktor 5 verbesserte Speichernutzung
und um den Faktor 50 verbesserte Startzeit gemessen werden, wenn statt des GraalVM HotSpot Mode GraalVM Native Image verwendet wird.

Dadurch eignet sich die Nutzung von GraalVM Native Image für die Nutzung von Java in Microservices und Serverless Cloud Functions, da hier bei jedem Start eines Services oder einer Funktion 
Kosten entsprechend der Startzeit und der Speichernutzung verursacht werden. 
