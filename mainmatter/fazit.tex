% !TEX root = main.tex
\newpage
\section{Fazit}
\label{sec:fazit}

Ziel dieser Arbeit war es einen Überblick über die NativeImage-Funktionalität von GraalVM zu geben, und deren Systemkomponenten und Funktionsweise zu erläutern. 
Durch \textit{native image} wird ein leichtgewichtige, in sich geschlossene ausführbare Datei (executable) erzeugt. Als Eingabe dient der Anwendungscode, dazugehörige Bibliotheken, das JDK und 
die notwendigen VM-Komponenten (SubstrateVM). Auf dieser Eingabe wird iterativ die Points-To Analyse, das Ausführen
von Initialisierungscode und das Heap-Snapshotting ausgeführt. Das Resultat wird anschließend kompiliert (AOT-Kompilierung) und in den Text-Abschnitt des executables geschrieben. Zudem wird ein Java-Heap, der \textit{image heap}
erstellt, welcher in den Data-Abschnitt des executables eingefügt wird. 

Durch die Initialisierung der Anwendung zur Build-Time und das Nutzen des \textit{image heaps} wird eine schnelle Startzeit, eine geringe Speichernutzung und eine mit der HotSpot JVM vergleichbare Spitzenleistung
ermöglicht.  Zur geringeren Speichernutzung trägt auch der Umstand bei, dass es nicht mehr notwendig ist die JVM während der Laufzeit laufen zu lassen,
 da keine Codeoptimierungen zur Laufzeit durch den JIT-Compiler und Profiler stattfinden. Dies trägt zudem zu einem konstanteren und besser vorhersagbarem Performanceverhalten der Anwendung bei.
 Beim Vergleich populärer Java-Microservice Frameworks konnte im Durchschnitt eine um den Faktor 5 verbesserte Speichernutzung
und um den Faktor 50 verbesserte Startzeit gemessen werden, wenn statt des GraalVM HotSpot Mode GraalVM Native Image verwendet wird.

Dadurch eignet sich die Nutzung von GraalVM Native Image für die Nutzung von Java in Microservices und Serverless Cloud Functions, da hier bei jedem Start eines Services oder einer Funktion 
Kosten entsprechend der Start-, Laufzeit und der Speichernutzung verursacht werden, und diese Faktoren somit, im Gegensatz zu maximalem Durchsatz, die höchste Priorität haben. 
Sobald allerdings der Heap Dimensionen von mehreren Gbyte - Tbyte annimmt und/oder Klassen bzw. andere Teile des Codes erst zur Laufzeit bekannt werden, und somit die Closed-World Annahme verletzen,
 eignet sich die Java HotSpotVM besser. 
