\subsection{Heap Snapshotting}
\label{subsec:heapsnapshotting}
Beim Heap Snapshotting wird aus dem \textit{type-flow graph} der Pointer Analyse ein Objekt-Graph generiert.

Die Wurzelzeiger (eng. root pointers) des Graphen sind statische Objekt Felder, die von der Pointer Analyse als \textit{read} markiert sind. Für jedes, als \textit{read} markierte, Objekt wird nun geprüft welche weiteren Objekte durch dessen Felder erreichbar sind. 
Dafür werden die Felder nachverfolgt, indem die Feldwerte durch Reflection gelesen werden. Dies ist möglich, da der \textit{image builder} eine Java Anwendung ist.
 Falls die Klasse des Feldwertes von der Pointer-Analyse noch nicht als möglicher Typ für das Feld ermittelt 
 wurde, wird die Klasse als weiterer Feldtyp des Feldes im \textit{type-flow graph} registriert (siehe Abbildung~\ref{fig:typeflowgraph}).

Nach dem Heap Snapshotting wird die Pointer-Analyse wieder ausgeführt, um den neuen Typ des Feldes allen transitiven Verwendungen des Feldes zu propagieren. Dies geschieht durch den \textit{type-flow graph}. 
Da im \textit{type-flow graph} neue Typen hinzugekommen sind, werden auch bei der ernuten Ausführung der Klasseninitialisierer neue Initialisierer ausgeführt. Anschließend wird wieder das Heap Snapshotting ausgeführt. 
Die iterative Ausführung dieser Phasen wird beendet, sobald die Pointer-Analyse in zwei aufeinanderfolgenden Iterationen keine Änderungen findet, wenn also sowohl die Klasseninitialisierer, als auch das Heap-Snapshotting keinen neuen Code erreichbar gemacht haben.
Die Objekte des Graphen werden serializiert und in die Data-Sektion des \textit{native image} geschrieben, und bilden somit den \textit{image heap}\parencite{Wimmer2019}.
